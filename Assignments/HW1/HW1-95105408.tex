\documentclass{article}
\usepackage[a4paper, top=1.5cm, bottom=2cm, left=1.5cm, right=1.5cm]{geometry}
\usepackage{url}
\usepackage{hyperref}
\usepackage{graphicx}
\usepackage{float}
\usepackage{framed}
\usepackage{multirow}
\usepackage{amsmath}
\usepackage[bottom]{footmisc}
\usepackage{xepersian}

\setlength{\parindent}{0pt}

\settextfont[Scale=1.25]{XB Niloofar}


\setdigitfont[Scale=0.9]{XB Niloofar}

\setlatintextfont[Scale=1.0]{Times New Roman}
\title{ حل تمارین ساختار و زبان کامپیوتر}
\author{سروش باسلی زاده\\95105408}
\begin{document}

\maketitle
\vspace{-8mm}
\hrulefill
\\[5mm]

\section{سوال اول}
\subsection{اندازه}
یکی از بزرگترین مزایای استفاده از ترانزیستورها اندازه کوچک تر آن هاست . گرچه لامپ های خلاء به مرور زمان کاهش اندازه داشتند و رفته رفته کوچک تر شدند ، ترانزیستورها  تا اندازه های به واقع میکروسکوپی کوچک شده اند. تفاوت اندازه بین ترانزیستورهاو لامپ های خلاء علت حرکت کامپیوترها از انداره هایی در حد یک اتاق به سمت اندازه های کامپیوترهای دسکتاپ امروزی بوده است. ترانزیستورها می توانند روزی تنها با استفاده از چند اتم ساخته شوند که این امر در ساخت چیپ های کامپیوتری که در آن ها از هزاران هزار ترانزیستور برای توان حداکثری کامپیوتر در یک چیپ استفاده می شود؛ کاربرد دارد.
\subsection{گرما}
ترانزیستور ترانزیستورها گرمای بسیار بسیار کمتری به نسبت لامپ های خلا تولید می کنند؛ این بدان معناست که آن ها می توانند در یک قطعه الکترونیکی با تراکم بسیار بیشتری قرار بگیرند و هم چنین نیازی به سیستم های سرمایشی \\(\lr{Cooling System}) بزرگ و پیچیده برای حفظ کارایی خود در طی یک مدت طولانی نداشته باشند.
\subsection{توان مصرفی}
ترانزیستورها هم چنین به نسبت لامپ های خلاء انرژی کمتری مصرف می کنند. این بدان معناست که آن ها به صرفه ترند و هزینه کمتری در استفاده دارند. این مورد هم چنین این اجازه را به مهندسین می دهد که در دستگاه هایی که با باتری کار می کنند طول مدت شارژدهی باتری را به طور قابل توجهی با استفاده از ترانزیستورها در سطح بیشتری نگاه دارند.
\subsection{دیرپایی}
ترانزیستورها به خاطر طراحی ساده و ساختار فلزی بسیار دیرپا هستند. لامپ های خلا از یک حباب شیشه ای استفاده می کنند که در خطر شکستن است و باید با دقت با آن کار شود. تنها ترانزیستورها مناسب قطعات الکترونیکی هستند که نیازمند تاب آوری در شرایط مشکل و کارکردن در شرایط محیطی دشوار هستند؛ می باشند. 


\section{سوال دوم}
\subsection{الف}
\subsubsection{کامپیوترهای کوانتومی}
به جای ذخیره اطلاعات با استفاده از صفرها و یک ها به مانند کامپیوترهای دیجیتالی ، کامپیوترهای کوانتومی از کوانتوم بیت (کیوبیت = \lr{qubit}) برای رمزگذاری اطلاعات استفاده می کنند بدین صورت که نشان دهنده صفرها یک ها و یا همزمان هردوی آن ها هستند. برهم نهی این حالات در کنار دیگر اصول مکانیک کوانتوم در رابطه با درهم تنیدگی کوانتومی و تونل زنی کوانتومی ، کامپیوترهای کوانتومی را قادر می سازد تا با ترکیبات عظیمی از حالات در یک آن کارکند.
\subsubsection{کامپیوترهای بیولوژیکی}
توضیح کامل در قسمت بعد (ب همین سوال)
\subsubsection{پردازش موازی یا توزیع‌شده}
به نحوی که محاسبات و پردازش‌های کامپیوتر، به چند قسمت تقسیم شده، و هر قسمت، سهمی از آن پردازش‌ها را داشته باشد.
\subsubsection{سخت افزارهای قابل بازپیکربندی}
 دارای جزئی انعطاف نرم‌افزار برای سرعت بالا در محاسبات است. تفاوت اصلی این سخت‌افزارها، تغییر مسیرهای داده در جریان کنترل است.
\subsection{ب}
در طی چند سال اخیر مشخص شده است که قانون مور به سمت توقف پیش می رود پس برای حفظ پیش روی مرزهای تکنولوژی کامپیوترها نیازمند آن هستیم که راجع به خود اجزا کامپیوترها دوباره اندیشی کنیم . حافظه دی ان ای ها می تواند یک راه حل در این زمینه باشد که میلیاردها میلیارد گیگابایت اطلاعات را که اینترنت را سرپا نگه میدارند کجا نگه داری کنیم ؟
\\
لوئیز چزه استاد علوم کامپیوتر در دانشگاه واشنگتن می گوید : « بخش اعظم اینکه چطور یک کامپیوتر بهتر بسازیم راجع به یافتن موادی بهتر برای ساخت کامپیوترها با آن ها هست. بنابراین، سیلیکون بنظر یک ماده بی نظیر برای این کار می اید اما اکنون به نقطه ای رسیده ایم که نامشخص است آیا می توانیم با سیلیکون به این مسیر ادامه بدهیم یا خیر ؟ پس من این موضوع که زیست شناسی بسیاری مولکول های کارا برای ساخت کامپیوترهای بهتر در اختیارمان می گذارد را حیرت انگیز می یابم. »
\\
تجهیرات ذخیره سازی اطلاعات کنونی مانند مرکز ذخیره سازی اطلاعات فیسبوک در اورجئون که به تازگی ساخته شده است تمامی انبارها را اشغال می کند و می تواند حدود یک اگزابایت را در بیشترین حالت ذخیره کند. این تنها کسری از اطلاعات تمام اینترنت است مقداری که حدود 
شانزده زتابایت یا همان شانزده هزار اگزابایت تا سال 2017 تخمین زده می شود.
\\
بارمزگذاری اطلاعات با استفاده از دی ان ای محققان ادعا میکنند میتوانند تمام آن اطلاعات را دراتاق پذیرایی شما جاکنند! با تبدیل بیت های اطلاعات از صفرها و یک ها از یک چیپ کامپیوتری به چهار حرف دی ان ای . دانشمندان می توانند رشته هایی از دی ان ای ها بسازند که هر اطلاعاتی را که بخواهیم ذخیره کنند از آهنگ ها تا اطلاعات کتابخانه ها.
\\
برای نیل به این هدف، محققان شاخصی (فهرستی) را می سازند که چهارنوکلئوتید سازنده دی ان ای \lr{(A،T،C و G)} را به رشته های صفر و یکی که هم اکنون بر روی کامپیوترهای خود داریم مرتبط می کند. یک سنتزگر دی ان ای رشته های کوتاه دی ان ای را که هریک بخش از کد فایل را نگه داری می کنند را می سازد هنگامی که تمام اطلاعات به دی ان ای تبدیل شدند می توان آن هارا با استفاده از یک ترتیب سنج دی ان ای که ترکیبهای مختلف نوکلئوتیدهارا می خواند؛ ذخیره و بازیابی کرد. \cite{discoverMag}
\\
\\
جدای از بحث حافظه های دی ان ای این کامپیوترهای در بحث پردازشی نیز کمک شایانی می کنند، انجام میلیون ها عملیات به صورت همزمان در این کامپیوترها ین امکان را فراهم می کند که سرعت کامپیوترهای براساس رشته های دی ان ای به صورت نمایی زیاد شود.  آزمایش لئونارد آدلمن با سرعت 1014 عملیات بر ثانیه ، 100 ترافلاپس اجرا شده بود در حالیکه سریع ترین ابر کامپیوتر دنیا با 35.8 ترافلاپس اجرا می شود. از طرفی اجرای موازی در این کامپیوترها پتانسیل افزایش سرعت محاسبات چندجمله ای و خیلی بزرگ (و البته حل پذیر) را با انجام عملیات کمتر را در خود دارد به عنوان مثال ترکیب 1018 تایی از رشته های دی ان ای می تواند تا 10000 برابر سوپرکامپیوترهای پیشرفته امروزی سریع باشد.\cite{stanfordEvaluation}
\\

\begin{thebibliography}{9}
\begin{LTRbibitems}
\resetlatinfont
\bibitem{discoverMag}
Scharping, N. (2016, April 6). DNA Data Storage Moves Beyond Moore’s Law. Retrieved from \url {http://blogs.discovermagazine.com/d-brief/2016/04/08/dna-data-storage/#.Wc35U2iCxPZ}
\bibitem{stanfordEvaluation}
Roberts, E and Hennesy, J. (2003). DNA Computing. Retrieved from \url {https://cs.stanford.edu/people/eroberts/courses/soco/projects/dna-computing/evaluation.htm}
\end{LTRbibitems}

\end{thebibliography}
\subsection{ج}

اینترنت اشیا شبکه دستگاه‌های فیزیکی، وسایل نقلیه و سایر اشیا نهفته با الکترونیک، نرم‌افزار، حسگرها و عملگرها و اتصال شبکه‌ای است، که آنها را به یکدیگر متصل کرده و آن‌ها را قادر به جمع‌آوری و تبادل اطلاعات می‌کند.
اینترنت اشیا این امکان را می‌دهد که اشیا از راه دور و توسط زیرساخت‌های شبکه موجود سنجش و مدیریت شوند. به‌وسیله چنین امکانی، فرصت مجتمع‌سازی دستگاه‌های فیزیکی به‌وجود آمده تحت سیستم‌های کامپیوتری ساخته می‌شود. چنین امری باعث افزایش بهره‌وری، دقت بالاتر و سود اقتصادی خواهد شد.\\
در ارتباط با جلوگیری از پایان فانون مور، باید در نظر داشت که اشیاء فیزیکی که به بکدیگر متصل‌اند، نه‌تنها دستگاه‌های در ارتباط با کاربر، که دستگاه‌های متصل به دستگاه‌های دیگر را هم شامل می‌شود. بدین ترتیب، و با اتصال مدارها و سخت‌افزارها،‌می‌توان اطلاعات را سریعتر و با مصرف انرژی کمتری ردوبدل کرد، که این امر منتج به کاهش استفاده از انرژی در واحد سطح خواهد شد.\\ادامه افزایش از اینترنت اشیاء و محبوبیت آن‌ها، باعث می‌شود که با ارتباط زیرساخت‌های کامپیوتر تحت شبکه به یکدیگر و با کاهش انرژی واحد سطح، از پایان قانون مور جلوگیری شود.\cite{iot} \cite{iot2}

\begin{thebibliography}{9}
\begin{LTRbibitems}
\resetlatinfont
\bibitem{iot}
Witeck, Ch. (2016, October 10). The internet of things is in your future - the law says so!. Retrieved from \url{http://internetofthingsagenda.techtarget.com/blog/IoT-Agenda/The-internet-of-things-is-in-your-future-the-law-says-so}
\resetlatinfont
\bibitem{iot2}
Internet Of Things. Retrived from \url{https://en.wikipedia.org/wiki/Internet_of_things}
\end{LTRbibitems}

\end{thebibliography}
\section{سوال سوم}
سه سال = 36 ماه \\
36 ماه / 18 ماه = 2 بار قانون مور\\
هر بار قانون مور -> دو برابر \\
پس $2^2$ برابر یعنی 4 برابر \\
\\
سرعت چهار برابر شده است پس اساسا در زمان مشابه می تواند برنامه ای با زمان مورد نیاز 4 برابر را اجرا کند، حال با توجه به رابطه زمان با سایز برنامه ها در هر مورد داریم :
\subsection{الف}
چون زمان 
O(n)
است و سرعت ما 4 برابر شده است پس برنامه ای با اندازه 4 برابر n را می تواند در همان زمان اجرا کند.

\subsection{ب}
چون زمان 
$O(n^2)$
است و سرعت 4 برابر شده است پس برنامه ای با اندازه 2 برابر n را می تواند در همان زمان اجرا کند.

\subsection{پ}
چون زمان 
$O(log{n})$
است و سرعت 4 برابر شده است پس برنامه ای با اندازه 
$n^4$
را می تواند در همان زمان اجرا کند.
\section{سوال چهارم}
با توجه به آنچه در کلاس در رابطه با \lr{ISA} مطرح شد (به عنوان مثال در مورد عدم تغییر معماری شرکت اینتل به منظور حفظ مشتریانی که نرم افزارهایشان با \lr{ISA} کنونی کار می کنند می توان گفت که :
{\lr{ISA}}یک واسط میان سخت افزار و نرم افزار است و ارتباط میان این دو توسط این بخش تامین می شود در نتیجه هنگام طراحی آن باید به نرم افزار نیز توجه شود بدین صورت که در حقیقت {\lr{ISA}} یک تفاهم میان سخت افزار و نرم افزار است و باید میان کارهایی که قرار است صورت گیرد میان این دو هماهنگی باشد مثلا آنکه قرار است چه {\lr{instruction}}هایی ایجاد شود و یا در کل اصلا قرار است چه {\lr{function}}هایی توسط این سیستم صورت گیرد برای همین باید به هماهنگی سخت افزار و نرم افزار توجه شود .
\\
به عنوان نمونه بر اثر تفاوت معماری پردازنده های کامپیوترهای دسکتاپ و گوشی های هوشمند و یا تبلت ها و تفاوت در {\lr{ISA}} ویندوز کامپیوترهای دسکتاپ قابل نصب بر روی گوشی های هوشمند نیست (گرچه مایکروسافت از زمان ویندوز 8 سعی در یکی کردن این دو دارد) این امر نشان از اهمیت توجه به نرم افزار مورد نیاز در هنگام طراحی {\lr{ISA}} دارد.
\section{سوال پنجم}

\subsection{الف}
سطح تجرید ({\lr{System Softwares}}) به طور دقیق تر سیستم عامل ({\lr{Operating Systems}})
\subsection{ب}
سطح تجرید برنامه های کاربردی ({\lr{Applications}})
\subsection{ج}
سطح تجرید سیستم عامل  ({\lr{Operating Systems}})
\subsection{د}
سطح تجرید ({\lr{System Softwares}}) به طور دقیق تر کامپایلر  ({\lr{Compilers}})
\subsection{ه}
سطح تجرید مدارات  ({\lr{Circuits}}) به طور دقیق تر  ({\lr{CMOS}})

\section{سوال ششم}
\subsection{الف}
سیستم های نهفته : سیستم هایی کامپیوتری با کاربردی خاص و از پیش محول شده هستنئد که یه عنوان بخشی از یک سیستم مکانیکی یا الکتریکی بزرگتر عمل می کنند این سیستم ها معمولا real-time هستند. این سیستم ها ویژگی هایی نظیر توان مصرفی کم، اندازه کوچک و قیمت بر واحد کم را دارا می باشند.

این سیستم ها در وسایل و سیستم های گوناگونی نظیر چراغ های راهنمایی و رانندگی، خودپردازهای بانکی، کنترل کننده های دما و لوازم های خانگی مانند ماشین های لباسشوئی، ظرفشوئی و ... استفاده می شوند.
برای توضیح بیشتر در یک ماشین لباسشوئی سیستم های نهفته کارهای گوناگونی مانند دریافت وزن لباس های گذاشته شده، بررسی اینکه آیا وزن انتخابی با نوع شستشو تطابق دارد یا خیر، تنظیم سطح شوینده و ... را انجام می دهند.
و یا در خودپردازها شمردن اسکناس ها بر اساس مبلغ انتخابی از انواع اسکناس ها ، خواندن و بررسی بارکد قبض های برق و ... و بررسی اطلاعات آن ها و ... را انجام می دهند.
\subsection{ب}
درگذشته گوشی های قدیمی به عنوان سیستم نهفته طبقه بندی می شدند زیرا اساسا دارای مجموعه ای از وظایف خاص و محول شده مانند ارسال پیامک، برقراری تماس و مواردی این چنینی را داشتند اما با عرضه گوشی های هوشمند این طبقه بندی مورد بحث و جدل واقع شده است.\\گروهی بر این باورند که از آنجا که این گوشی ها قادرند وظایف از پیش تعیین نشده (محول نشده) ای نظیر اجرای اپلیکیشن های متفاوت را برعهده گیرند در این طبقه بندی نمی گنجند آن ها همچنین با زیر سوال بردن اینکه این گوشی ها دیگر ویژگی های سیستم های نهفته مانند اندازه ، توان مصرفی و قیمت را نیز دارا نمی باشند و هر لحظه بیشتر از این ویژگی ها فاصله می گیرند و وظایف مختلف غیر محول شده ی گوناگون بیشتری را انجام می دهند قرارگیری این گوشی ها در این طبقه بندی را مردود می دانند.\\اما گروهی دیگر بر این باورند که گوشی های هوشمند هم چنان باید در همین دسته طبقه بندی شوند چرا که هنوز نمی توان آن ها را سیستم هایی با هدف کاربری عمومی (\lr{General-Purposed}) دانست (به مانند کامپیوترهای دسکتاپ) و هم چنین بسیاری از عملکرد های آن ها همان عملکردهای از پیش محول شده کارخانه ای هستند مثلا کارهایی نظیر اتصال به اینترنت ، دوربین و ... کارهای از پیش تعیین شده اند و صرف اجرای اپلیکیشن هار باعث نمی شود آنها کامپیوترهای \lr{General-Purposed} باشند این تنها بخش کوچکی از عملکرد کامپبوترهایی \lr{General-Purposed} مانند کامپیوترهای دسکتاپ است.

\end{document}